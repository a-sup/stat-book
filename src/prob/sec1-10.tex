\section{Совокупности случайных величин. Совместная функция распределения. Независимость случайных величин. Критерии независимости. Ковариация, коэффициент корреляции}

\subsubsection{Совместное распределение, его свойства}

Пусть случайные величины $\xi_1, \ldots, \xi_n$ заданы на одном вероятностном пространстве $(\Omega, \SigAlg \MyPr)$.
\begin{defn}
    \textit{Совместное распределение} случайных величин $(\xi_1, \ldots, \xi_n)$~--- функция $\MyPr\colon \Borel(\Real^{n}) \to \Real$:
    \begin{equation*}
        \MyPr(\xi \in B) = \MyPr(\omega \colon (\xi_{1}(\omega), \ldots, \xi_{n}(\omega)) \in B),~ B \subset \Borel(\Real^{n}).
    \end{equation*}
\end{defn}
\begin{defn}
    \textit{Функция совместного распределения} случайных величин $(\xi_1, \ldots, \xi_n)$~--- функция $F\colon \Real^{n} \to \Real$:
    \begin{equation*}
        F(x_{1}, \ldots, x_{n})=\MyPr(\xi_{1}<x_{1}, \ldots, \xi_{n}<x_{n}).
    \end{equation*}
\end{defn}

\begin{rmrk}
    Для функции совместного распределения выполняются свойства, аналогичные одномерному случаю~--- сохраняется монотонность, непререрывность слева по каждой переменной.
    При этом частные функции распределения восстанавливаются по совместной следующим образом:
    \begin{equation*}
        \lim\limits_{\substack{x_{k} \to +\infty \\ k \neq i}}  F(x_{1}, \ldots, x_{i}, \ldots, x_{n}) = F_{i}(x), \quad i = \overline{1,n}.
    \end{equation*}
    Следует, однако, обратить внимание на предельные свойства. 
    Для простоты проиллюстрируем их на двумерном случае.
    \begin{align*}
        \lim\limits_{x_1 \to -\infty} F(x_1, x_2) & = 0 \\
        \lim\limits_{x_2 \to -\infty} F(x_1, x_2) & = 0 \\
        \lim_{
            \substack{
                x_1 \to +\infty \\ 
                x_2 \to +\infty
            }
        } F(x_1, x_2) & = 1
    \end{align*}
    В самом деле, по определению $F(x_1, x_2) = \MyPr(\xi_1 < x_1, \xi_2 < 2) = \MyPr(B_1 \cap B_2),$ 
    где $B_1 = \{\xi_1 < x_1\}, B_2 = \{\xi_2 < x_2\}$. 
    Если хоть одно из событий "<стремится"> к пустому множеству (что и происходит при $x_1 \to -\infty$ или $x_2 \to -\infty$), 
    то и пересечение делает то же самое.

    Если же мы хотим получить в пересечении $\Omega$ (и, следовательно, вероятность~$1$), то необходимо устремить $x_1$ и $x_2$ к $+\infty$ \textit{одновременно}.
\end{rmrk}

Далее рассматриваем совместные распределения двух случайных величин.

\subsubsection{Виды многомерных распределений}

\begin{defn}
    Случайные величины $\xi_1, \xi_2$ имеют дискретное совместное распределение, если существует не более чем счётный набор пар чисел $\{a_{i}, b_{j}\}$ такой, что
    \begin{equation*}
        \sum\limits_{i=1}^{\infty} \sum\limits_{j=1}^{\infty} \MyPr\left(\xi_{1}=a_{i}, \xi_{2}=b_{j}\right)=1.
    \end{equation*}
    Таблицу, на пересечении $i$-й строки и $j$-го столбца которой стоит вероятность $\MyPr\left(\xi_{1}=a_{i}, \xi_{2}=b_{j}\right)$, называют \textit{таблицей совместного распределения} случайных величин $\xi_1$ и $\xi_2$.
\end{defn}
\begin{defn}
    Случайные величины $\xi_1, \xi_2$ имеют абсолютно непрерывное совместное распределение, если существует неотрицательная функция $f_{\xi_{1}, \xi_{2}}(x, y)$ такая, что для любого борелевского множества $B \in \Borel\left(\Real^{2}\right)$ имеет место равенство
    \begin{equation*}
        \MyPr\bigl(\left(\xi_{1}, \xi_{2}\right) \in B\bigr)=\iint\limits_{B} f_{\xi_{1}, \xi_{2}}(x, y) \, dx dy.
    \end{equation*}
    Если такая функция $f_{\xi_{1}, \xi_{2}}(x, y)$ существует, она называется \textit{плотностью совместного распределения} случайных величин $\xi_1, \xi_2$.
    
    Функция совместного распределения в этом случае имеет вид:
    \begin{equation*}
        F(x, y)=\MyPr(\xi_{1}<x, \xi_{2}<y)=\int\limits_{-\infty}^{x}\left(\int\limits_{-\infty}^{y} f_{\xi_{1}, \xi_{2}}(u, v) \, dv\right) du.
    \end{equation*}
\end{defn}

\begin{rmrk}
    Плотность совместного распределения имеет те же свойства, что и плотность распределения одной случайной величины: неотрицательность и нормированность:
    \begin{equation*}
        f(x, y) \geqslant 0~ \forall x, y \in \Real, \quad \iint\limits_{\Real^{2}} f(x, y) \, dx dy = 1.
    \end{equation*}

    По функции совместного распределения его плотность находится как смешанная частная производная (в точках, где она существует):
    \begin{equation*}
        f(x, y)=\frac{\partial^{2}}{\partial x \partial y} F(x, y).
    \end{equation*}
\end{rmrk}

\begin{rmrk}
    Из существования плотностей $\xi_1$ и $\xi_2$ не следует абсолютная непрерывность совместного распределения этих случайных величин. 
    Например, вектор $(\xi, \xi)$ принимает значения только на диагонали в $\Real^2$ и уже поэтому не имеет плотности распределения (его распределение сингулярно). 
    Обратное же свойство, как показывает следующая теорема, всегда верно.
\end{rmrk}

\begin{thm*}
    Если случайные величины $\xi_1$ и $\xi_2$ имеют абсолютно непрерывное совместное распределение с плотностью $f(x, y)$, то $\xi_1$ и $\xi_2$ в отдельности также имеют абсолютно непрерывное распределение с плотностями:
    \begin{equation*}
        f_{\xi_{1}}(x)=\int\limits_{-\infty}^{\infty} f(x, y) dy, \quad f_{\xi_{2}}(y)=\int\limits_{-\infty}^{\infty} f(x, y) dx.
    \end{equation*}
    Для $n > 2$ плотности случайных величин $\xi_1, \ldots, \xi_n$ находятся по плотности их совместного распределения $f(x_1, \ldots, x_n)$ интегрированием функции $f$ по всем <<лишним>> координатам.
\end{thm*}
\begin{proof}
\begin{equation*}
    F_{\xi_{1}}\left(x_{1}\right)
    = \lim _{x_{2} \to+\infty} F_{\xi_{1}, \xi_{2}}\left(x_{1}, x_{2}\right)
    = \int\limits_{-\infty}^{x_{1}}\left(\int\limits_{-\infty}^{\infty} f(x, y) d y\right) d x
    = \int\limits_{-\infty}^{x_{1}} f_{\xi_{1}}(x) d x
\end{equation*}
\end{proof}

\subsubsection{Независимость случайных величин}
\begin{defn}
    Случайные величины $\xi_1, \ldots, \xi_n$ называют \textit{независимыми в совокупности}, если для любого набора борелевских множеств $B_{1}, \ldots, B_{n} \in \Borel(\Real)$:
    \begin{equation*}
        \MyPr\left(\xi_{1} \in B_{1}, \ldots, \xi_{n} \in B_{n}\right)=\MyPr\left(\xi_{1} \in B_{1}\right) \cdot \ldots \cdot \MyPr\left(\xi_{n} \in B_{n}\right)
    \end{equation*}
\end{defn}
\begin{namedthm}[Критерий независимости]
    Случайные величины $\xi_1, \ldots, \xi_n$ независимы в совокупности $\iff$ имеет место равенство:
    \begin{equation*}
        F_{\xi_{1}, \ldots, \xi_{n}}\left(x_{1}, \ldots, x_{n}\right)=F_{\xi_{1}}\left(x_{1}\right) \cdot \ldots \cdot F_{\xi_{n}}\left(x_{n}\right).
    \end{equation*}
    В частности, в случае дискретного совместного распределения:
    \begin{equation*}
        \MyPr\left(\xi_{1}=a_{1}, \ldots, \xi_{n}=a_{n}\right)=\MyPr\left(\xi_{1}=a_{1}\right) \cdot \ldots \cdot \MyPr\left(\xi_{n}=a_{n}\right) \quad \forall a_1, \ldots, a_n \in \Real.
    \end{equation*}
    В случае абсолютно непрерывного:
    \begin{equation*}
        f_{\xi_{1}, \ldots, \xi_{n}}\left(x_{1}, \ldots, x_{n}\right)=f_{\xi_{1}}\left(x_{1}\right) \cdot \ldots \cdot f_{\xi_{n}}\left(x_{n}\right).
    \end{equation*}
\end{namedthm}

\subsubsection{Формула свёртки}
Пусть $\xi_1, \xi_2$~--- случайные величины с плотностью совместного распределения $f_{\xi_{1}, \xi_{2}}\left(x_{1}, x_{2}\right)$, задана борелевская функция $g\colon \Real^{2} \to \Real$. 
Требуется найти функцию распределения (и плотность, если она существует) случайной величины $\eta=g\left(\xi_{1}, \xi_{2}\right)$.
\begin{lem}
    Пусть $x \in \Real$, задана область $D_{x} \subseteq \Real^{2},~ D_x = \{(u,v)\colon g(u,v) < x\}$ Тогда случайная величина $\eta=g\left(\xi_{1}, \xi_{2}\right)$ имеет функцию распределения
    \begin{equation*}
        F_{\eta}(x)=\MyPr\left(g\left(\xi_{1}, \xi_{2}\right)<x\right)=\MyPr\left(\left(\xi_{1}, \xi_{2}\right) \in D_{x}\right)=\iint\limits_{D_{x}} f_{\xi_{1}, \xi_{2}}(u, v) du dv.
    \end{equation*}
\end{lem}
Далее считаем, что случайные величины $\xi_1$ и $\xi_2$ независимы, т. е. $f_{\xi_{1}, \xi_{2}}(u, v) \equiv f_{\xi_{1}}(u) f_{\xi_{2}}(v)$. В этом случае распределение величины $g\left(\xi_{1}, \xi_{2}\right)$ полностью определяется частными распределениями величин $\xi_1$ и $\xi_2$.
\begin{namedthm}[Формула свёртки]
    Если случайные величины $\xi_1$ и $\xi_2$ независимы и имеют абсолютно непрерывные распределения с плотностями $f_{\xi_{1}}(u)$ и $f_{\xi_{2}}(v)$, то плотность распределения суммы $\xi_{1}+\xi_{2}$ существует и равна <<свёртке>> плотностей $f_{\xi_{1}}$ и $f_{\xi_{2}}$:
    \begin{equation*}
        f_{\xi_{1}+\xi_{2}}(t)=\int\limits_{-\infty}^{\infty} f_{\xi_{1}}(u) f_{\xi_{2}}(t-u) du=\int\limits_{-\infty}^{\infty} f_{\xi_{2}}(u) f_{\xi_{1}}(t-u) du.
    \end{equation*}
\end{namedthm}

\begin{proof}
    Воспользуемся утверждением вышеуказанной леммы для борелевской функции $g(u, v)=u+v$. 
    Интегрирование по двумерной области $D_{x}=\{(u, v) \colon u+v<x\}$ можно заменить последовательным вычислением двух интегралов: 
    наружного — по переменной $u$, меняющейся в пределах от $-\infty$ до $+\infty$, и внутреннего~--- по переменной $v$, которая при каждом $u$ должна быть меньше, чем $x-u$. 
    Поэтому
    \begin{equation*}
        F_{\xi_{1}+\xi_{2}}(x)=\iint\limits_{D_{x}} f_{\xi_{1}}(u) f_{\xi_{2}}(v) dv du=\int\limits_{-\infty}^{\infty}\left(\int\limits_{-\infty}^{x-u} f_{\xi_{1}}(u) f_{\xi_{2}}(v) dv\right) du.
    \end{equation*}
    
    Сделаем в последнем интеграле замену $v=t-u$. При этом $v \in(-\infty, x-u) \iff t \in(-\infty, x), d v=d t$. 
    В полученном интеграле меняем порядок интегрирования:
    \begin{equation*}
        F_{\xi_{1}+\xi_{2}}(x)=\int\limits_{-\infty}^{\infty} \int\limits_{-\infty}^{x} f_{\xi_{1}}(u) f_{\xi_{2}}(t-u) d t d u=\int\limits_{-\infty}^{x}\left(\int\limits_{-\infty}^{\infty} f_{\xi_{1}}(u) f_{\xi_{2}}(t-u) du\right) dt.
    \end{equation*}
    Из функции распределения $F_{\xi_{1}+\xi_{2}}(x)$ выражается плотность $f_{\xi_{1}+\xi_{2}}(t)$.
\end{proof}

\subsubsection{Ковариация, коэффициент корреляции, их свойства}

\begin{defn} Рассмотрим случайные величины $\xi$ и $\eta$. 
Дисперсия их суммы в общем случае равна $\Var(\xi+n)=\Var \xi+\Var \eta + 2 \bigl( \Exp (\xi \eta)-\Exp \xi \, \Exp \eta \bigr)$. Величина $\operatorname{cov}(\xi, \eta) \equiv \Exp (\xi \eta)-\Exp \xi \, \Exp \eta$ называется \textit{ковариацией} случайных величин $\xi$ и $\eta$. Если $\xi$ и $\eta$ независимы, то $\operatorname{cov}(\xi, \eta) = 0$. Обратное, вообще говоря, неверно.
\end{defn}

\hypertarget{counter_exmp_independence}{}
\begin{exmp}
    Рассмотрим $\xi \sim \Uniform_{[-\pi;\pi]}$, случайные величины ${\eta_{1}=\cos \xi}$ и ${\eta_{2}=\sin \xi}$.
    \begin{enumerate}
        \item 
            Докажем некореллированность данных случайных величин.
            \begin{gather*}
                \Exp \eta_{1}=\int\limits_{-\pi}^{\pi} \cos x \cdot \frac{1}{2 \pi} d x=0, \quad \Exp \eta_{2}=\int\limits_{-\pi}^{\pi} \sin x \cdot \frac{1}{2 \pi} d x=0 \\
                \Exp \eta_{1} \eta_{2}=\int\limits_{-\pi}^{\pi}(\cos x \sin x) \frac{1}{2 \pi} d x=\frac{1}{4 \pi} \int\limits_{-\pi}^{\pi} \sin 2 x d x=0
            \end{gather*}
            Следовательно, $\operatorname{cov}(\eta_1, \eta_2) = 0$.
        \item 
            Докажем зависимость $\eta_1$ и $\eta_2$. Рассмотрим события:
            \begin{equation*}
                A = \left\{\omega \colon \eta_1(\omega) \in \left[0, \frac{1}{2} \right] \right\}, \quad
                B = \left\{\omega \colon \eta_2(\omega) \in \left[0, \frac{1}{2} \right] \right\},
            \end{equation*}
            Проверим по критерию независимости:
            \begin{gather*}
                \MyPr\left\{\eta_{1} \in\left[0, \frac{1}{2}\right]\right\}=\MyPr\left\{\xi \in\left[-\frac{\pi}{2},-\frac{\pi}{3}\right] \cup\left[\frac{\pi}{3}, \frac{\pi}{2}\right]\right\}=\frac{1}{2 \pi} \cdot 2 \cdot \frac{\pi}{6}=\frac{1}{6} \\
                \MyPr\left\{\eta_{2} \in\left[0, \frac{1}{2}\right]\right\}=\MyPr\left\{\xi \in\left[0, \frac{\pi}{6}\right] \cup\left[\frac{5 \pi}{6}, \pi\right]\right\}=\frac{1}{2 \pi} \cdot 2 \cdot \frac{\pi}{6}=\frac{1}{6} \\
                \MyPr\left\{\eta_{1} \in\left[0, \frac{1}{2}\right], \eta_{2} \in\left[0, \frac{1}{2}\right]\right\}=\MyPr\{\varnothing\}=0 \neq \frac{1}{6} \cdot \frac{1}{6}
            \end{gather*}
            Следовательно, $\eta_1$ и $\eta_2$ зависимы.
    \end{enumerate}
\end{exmp}

\begin{namedthm}[Свойства ковариации]\leavevmode
    \begin{enumerate}
        \item 
            $\operatorname{cov}(\xi, \xi)=\Var \xi$;
        \item 
            $\operatorname{cov}(\xi, \eta)=\operatorname{cov}(\eta, \xi)$;
        \item 
            $\forall a, b \in \Real ~ \operatorname{cov}(a \xi + b, \eta)=a \operatorname{cov}(\xi, \eta)$;
        \item 
            $\operatorname{cov}(\eta + \zeta, \xi) = 
            \operatorname{cov}(\eta, \xi) + \operatorname{cov}(\zeta, \xi)$;
        \item 
            $\operatorname{cov}^2(\xi, \eta) \leqslant \Var\xi \, \Var\eta$,
        
            $\operatorname{cov}^2(\xi, \eta) = \Var\xi \,\Var\eta \iff \xi \overset{\text{п.н.}}{=} a\eta + b, \; a, b \in \Real$ (аналог неравенства Коши-Буняковского);
    \end{enumerate}
\end{namedthm}

Величина ковариации характеризует меру линейной зависимости случайных величин. Однако от умножения на константу (не равную нулю) зависимость случайных величин не изменяется, в отличие от ковариации. Введём новый термин.

\begin{defn}
    \textit{Коэффициент корреляции} $\rho(\xi,\eta)$ случайных величин $\xi$ и $\eta$, дисперсии которых существуют и отличны от нуля:
    \begin{equation*}
        \rho(\xi, \eta)=\frac{\operatorname{cov}(\xi, \eta)}{\sqrt{\Var \xi} \sqrt{\Var \eta}}.
    \end{equation*}
\end{defn}

\begin{namedthm}[Свойства коэффициента корреляции]\leavevmode
    \begin{enumerate}
        \item 
            Коэффициент корреляции независимых случайных величин равен нулю;
        \item 
            Для любых двух случайных величин (для которых выполнены условия определения) их коэффициент корреляции по модулю не превосходит единицы;
        \item 
            Если $\bigl| \rho(X,Y) \bigr| = 1$, то с вероятностью один $X$ и $Y$ линейно выражаются друг через друга:
            \begin{equation*}
                \bigl| \rho(X, Y) \bigr| = 1 \Longrightarrow \exists a \neq 0, \, b \in \Real\colon \MyPr(X - aY = b) = 1.
            \end{equation*}
            При этом знак коэффициента $a$ совпадает со знаком коэффициента корреляции.
    \end{enumerate}
\end{namedthm}

\begin{proof}
    \begin{enumerate}
    \item 
        В числителе дроби, которой равен коэффициент корреляции,
        окажется ноль. В знаменателе нуля быть не должно, это обеспечивается определением.

    \item 
        Обозначим эти две случайные величины как $\xi$ и $\eta$ и центрируем: $\xi_c = \xi - \Exp \xi$ и $\eta_c = \eta - \Exp \eta$. 
        Так как $\operatorname{cov}(\xi, \eta)=\operatorname{cov}\left(\xi_{c}, \eta_{c}\right)$, а дисперсия случайной величины не меняется от смещения случайной величины на константу, коэффициент корреляции не изменится.
        
        Далее, т.к. $\Exp \xi_{c}=\Exp \eta_{c}=0$:
        \begin{gather*}
            \Var \xi_{c}=\Exp \xi_{c}^{2}-\left(\Exp \xi_{c}\right)^{2}=\Exp \xi_{c}^{2},~ \Var \eta_{c}=\Exp \eta_{c}^{2}; \\
            \operatorname{cov}\left(\xi_{c}, \eta_{c}\right)=\Exp \left(\xi_{c} \, \eta_{c}\right) - \Exp \xi_{c} \, \Exp \eta_{c}=\Exp \left(\xi_{c} \, \eta_{c}\right).
        \end{gather*}
        
        Далее идут те же рассуждения, что часто используются при доказательстве неравенства Коши-Буняковского:
        \begin{equation*}
            \forall a \in \Real \quad 0 \leqslant \Var\left(\xi_{c}-a \eta_{c}\right) = 
            \Exp \left(\xi_{c} - a \eta_{c}\right)^{2} - \bigl(\Exp \left(\xi_{c}-a \eta_{c}\right)\bigr)^{2} = 
            \Exp \left(\xi_{c}-a \eta_{c}\right)^{2}.
        \end{equation*}
        
        Полученное неравенство можно рассматривать как квадратное неравенство относительно $a$, а именно
        \begin{equation*}
            \Exp \left(\xi_{c}-a \eta_{c}\right)^{2}=\Exp \xi_{c}^{2}-2 a \Exp \left(\xi_{c} \eta_{c}\right)+a^{2} \Exp \eta_{c}^{2} \geqslant 0.
        \end{equation*}
        
        Поскольку верно это для любого $a$, то дискриминанту нельзя быть больше нуля. То есть:
        \begin{multline*}
            \bigl(\Exp \left(\xi_{c} \eta_{c}\right) \bigr)^{2} - \Exp \xi_{c}^{2} \, \Exp \eta_{c}^{2} \leqslant 0 \Longleftrightarrow \bigl|\Exp \left(\xi_{c} \eta_{c}\right)\bigr| \leqslant \sqrt{\Exp \xi_{c}^{2} \, \Exp \eta_{c}^{2}} \implies \\
            \implies \bigl|\operatorname{cov}\left(\xi_{c}, \eta_{c}\right)\bigr| \leqslant \sqrt{\Var \xi_{c} \, \Var \eta_{c}}.
        \end{multline*}
        
        По доказанному выше <<стирание>> индексов не изменит коэффициентов.

    \item 
        Доказательство этого свойства целиком опирается на доказательство предыдущего: если выполнилось равенство $|\operatorname{cov}(\xi, \eta)|=\sqrt{\Var \xi \, \Var \eta}$, 
        то квадратное неравенство относительно $a$ обратилось в равенство.
        Но это равенство означает, что равна нулю $\Var(\xi - a \eta)$, а это сразу говорит о том, что с вероятностью один $\xi - a\eta$ равна константе.
        Обозначим эту константу за $b$ и получим то, что нужно было доказать.
        
        Знак коэффициента корреляции совпадает с знаком ковариации, так дисперсии по предположению положительны. Выразив $\xi$ через $\eta$, мы можем воспользоваться свойствами ковариации и получить
        \begin{equation*}
            \rho(\xi, \eta) = \rho(a\eta + b, \eta)=
            \frac{\operatorname{cov}(a\eta + b, \eta)}
            {\sqrt{\Var(a\eta + b) \, \Var\eta}} = 
            \frac{a\operatorname{cov}(\eta, \eta)}
            {\sqrt{a^2\, \Var\eta \, \Var\eta}} = 
            \frac{a\Var\eta}
            {|a|\Var\eta} = 
            \text{sign}(a).
        \end{equation*}

\end{enumerate}
\end{proof}
